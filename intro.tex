
\chapter{Wstęp: znaczenie niezawodnej infrastruktury sieciowej}

Wstęp do pracy.

\section{Zastosowanie testów sieci i urządzeń sieciowych}
\subsection{Dlaczego warto testować infrastrukturę}

Sprawdzenie możliwości architektury\\
Symulacja ataków (pentesty)\\
Poznanie realnej wydajności infrastruktury


\subsection{Dlaczego warto testować urządzenia sieciowe}

Zgodność ze specyfikacją \\
Szukanie podatności w urządzeniach \\
Rola testów przy zakupach (nowych inwestycjach ) - spełnienie wymagań projektowych


\section{Wprowadzenie do wysokiej dostępności (ang. \gls{HA}) i równoważenia obciążenia (ang. \gls{LB})}


\info{(ten rozdział jest ponieważ praca dotyczy również testowania architektury, w testach przewidziany jest load-balancing, zob. schematy labu)}

Co to jest wysoka dostępność i dlaczego to robimy, SPoF

Cel uzyskania niezawodnej i optymalnie wykorzystanej architektury


\subsection{Algorytmy \gls{LB}}

\subsection{Znaczenie session-persistence}

\subsection{Przykłady}

\begin{description}
\item[Alteon VADC] asd
\item[HAProxy] asd
\item[Keepalived + pacemaker] asd
\end{description}