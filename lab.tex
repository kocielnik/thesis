\chapter{Metodyka testów i pomiarów}

 \info{tutaj dobrze by było, żeby wybrane zostały testy reprezentujące różne grupy ataków DDoS klasyfikacji wprowadzonej w 4.4}
\section{Testy wolumetryczne}
 
6.1.1.	 Ataki proste typu TCP-SYN flood,  HTTP(S) flood, ICMP flood, TCP OOS, bądź mix
6.1.2.	 Ataki typu reflected, DNS, NTP – wymagają zbudowania bądź zasymulowania architektury
6.1.3.	 Zawsze jakiś procent ruchu jest ruchem legalnym
\section{Analiza testów wolumetrycznych}

6.2.1.	 Poprawność mitygacji (czy ruch legalny dociera, albo jaki %)
6.2.2.	 Czas mitygacji od rozpoczęcia ataku
6.2.3.	 Pojemność ruchowa mitygacji i ruchu legalnego w stosunku do dokumentacji, bądź wartości teoretycznej, mierzona w Gbps, PPS
6.2.4.	 Pojemność jednoczesnych sesji HTTP, SSL
6.2.5.	 Pomiar opóźnień jeżeli mitygacja jest aktywna
6.2.6.	 
6.3.	Testy fuzzingowe* - jeżeli starczy czasu
6.3.1.	 DefensePro ma Anomaly Engine, można sprawdzić


\chapter{Architektura labu}

7.1.	Radware DefensePro w trybie inline 
7.2.	Radware DefenseFlow (BGP FlowSpec)
7.3.	Radware DefensePro + DefenseFlow (sygnalizacja własnościowa)
7.4.	Arbor Networks
(brak rysunku)

7.5.	FastNetMon + HAProxy (opensource only) 
  https://fastnetmon.com/

7.6.	Flowmon* - jest darmowy eval

\section{Widok trasy jednego requestu (tylko w jedną stroną) w warstwie L2}

W górnym rzędzie znajdują się hosty wirtualizowane (najprawdopodobniej ESXi), w dolnym rzędzie applience fizyczne. Pokazuje to z jakim overhead mamy do czynienia. 
Koszt trasy pomiędzy hostami na wirtualizatorze jest możliwy do pominięcia (operacje w RAMie), natomiast link pomiędzy ESXi i appliancami trzeba pomyśleć o jakimś LACPie może
