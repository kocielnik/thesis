
\chapter{Wprowadzenie do problematyki czegośtam}
\label{chapter:background}

Tekst :)

\section{Sekcja X}

...

\section{Sekcja Y}

...

\subsection{Podsekcja A}

...

\subsection{Podsekcja B}

...

\section{Sekcja Z}

...

\subsection{Podsekcja C}

...

\begin{figure}[ht]
	\begin{center}
		\includegraphics[width=5cm]{foto}
	\end{center}
	\caption{Obrazek}
	\label{etykieta_obrazka}
\end{figure}

Opis, opis, na rysunku~\ref{etykieta_obrazka} jest pingwin!

\subsection{Podsekcja D}

Jakiś tekst. W pracy X~\cite{praca_x} opisano cośtam.
A w rozdziale~\ref{chapter:background} coś podobnego ;)

\section{Sekcja W}
\label{sec:sekcja_w}

...
Ta sekcja zawiera \texttt{coś ciekawego}, oraz \textit{coś jeszcze}.

\begin{description}
  \item[1] Element \texttt{A}
  \item[2] Element \texttt{B}
  \item[3] Element \texttt{C}
\end{description}

\section{Sekcja V}
\label{sec:sekcja_v}

Tabelka:
\begin{center}
  \begin{tabular}{ | l || c | c | }
    \hline
    \multirow{2}{*}{Algorytm} & \multicolumn{2}{|c|}{Czas symulacji [sek]} \\
     & implementacji X & implementacji Y \\
    \hline \hline
    A & 15 & 15 \\
    \hline
    B & 10 & 15 \\
    \hline
    C & 12 & 13 \\
    \hline
    D & 110 & 230 \\
    \hline
  \end{tabular}
\end{center}

