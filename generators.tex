\chapter{Przegląd generatorów ruchu sieciowego}
\section{Generatory sprzętowe}

charakterystyka, przykłady
\begin{itemize}
\item Spirent
\item Ixia
\end{itemize}





\section{Generatory programowe w Linuksie}

\subsection{Metody generowania wielkowolumenowego ruchu}

Opis procesu generowania pojedynczego pakietu w Linuksie \\
Po co robić memory zero-copy \\
Szybkie vs wolne backendy: SOCKET\_RAW, libpcap, netmap, PF\_RING, AF\_PACKET \\

\subsection{Funkcjonalności różnych generatorów/frameworków}
Badanie: netsniff-ng, scapy, PKTGEN)

\subsection{Wyspecjalizowane generatory}
 JMeter
\subsection{Fuzzery}

Na tą chwilę brak wiedzy/doświadczenia z tego typu programami




\section{Generatory – analiza komparatywna}

 (tabelka zalety-wady)
 
	Funkcjonalność vs Wydajność vs Cena
	
\begin{center}
  \begin{tabular}{ | l || c | c | c |}
    \hline
    Metoda & Funkcjonalność & Wydajność & Cena \\
    \hline \hline
    A & 15 & 15 & 1\\
    \hline
    B & 10 & 15 & 2 \\
    \hline
    C & 12 & 13 & 3\\
    \hline
    D & 110 & 230 & 4\\
    \hline
  \end{tabular}
\end{center}

\section{Metody analizy ruchu sieciowego}
\improvement{lepsza klasyfikacja}
\subsection{Metody opierające się na (kopii) ruchu}
\subsubsection{Urządzenie in-line}
\paragraph{Kopia ruchu (port-mirroring)}
\paragraph{Backendy: netmap, PF\_RING, pcap}
\subsubsection{Metody statystyczne}
\paragraph{Flowy: sFlow, NetFlow}
\paragraph{SNMP/Netconf}
\subsection{Analiza komparatywna (tabelka)}

\begin{center}
  \begin{tabular}{ | l || c | c | }
    \hline
    \multirow{2}{*}{Algorytm} & \multicolumn{2}{|c|}{Czas symulacji [sek]} \\
     & implementacji X & implementacji Y \\
    \hline \hline
    A & 15 & 15 \\
    \hline
    B & 10 & 15 \\
    \hline
    C & 12 & 13 \\
    \hline
    D & 110 & 230 \\
    \hline
  \end{tabular}
\end{center}

